\section{Architecture}
\label{section:architecture}

\begin{figure*}[t]
    \includesvg[width=\textwidth]{GESTALT_Diagram3.svg}
    \centering
    \caption[width=\textwidth]{\emph{GESTALT} consists of a data collection module, an ownership assignment module, a concept mapping module, and a search module.}
    \label{fig:architecture}
\end{figure*}

%The Architecture of \textit{GESTALT} is designed to be lightweight and modular. The core requirement is to automate the last-mile search problem and allow users to search for locations of interest based on partial information. 
%Achieving this outcome necessitates collecting and processing objects and locations autonomously, at scale.
%Given a large number of noisy object tags, \emph{GESTALT} performs fuzzy object assignment, allowing objects to be assigned to multiple nearby locations, thereby improving the system's recall.
%The subsequent concept mapping and spatial search processes maintain the system's precision using the relative directional relationships between query objects, extracting implicit information about the user's target location. 

The architecture of \textit{GESTALT} (Figure \ref{fig:architecture}) covers four essential functions needed to perform last-mile search: data acquisition, object ownership assignment, concept mapping, and search. 
We describe and motivate each of them below, leaving the implementation details for sections \ref{section:data} through \ref{section:search}. 

\textbf{Data Acquisition.}
The Data Acquisition subsystem of \emph{GESTALT} leverages a variety of data sources to ingest or tag objects and locations, including hand-labeled data in Keyhole Markup Language~\footnote{\href{https://developers.google.com/kml/documentation/kml\_tut}{https://developers.google.com/kml/documentation/kml\_tut}} (KML) files, crowd-sourced object and location labels from Open Street Maps (OSM)~\cite{Haklay2008}, and automatically generated object tags extracted from Flickr images geolocated within the region of interest. 
Data Acquisition fuses heterogeneous data formats to enable common representation, storage and search.

\textbf{Ownership Assignment.}
The Ownership Assignment subsystem of \emph{GESTALT} maps the \textit{objects} to the \textit{locations} they most likely belong to using fuzzy clustering that allows objects to be assigned to multiple locations. 
\emph{GESTALT}'s assignment protocol accounts for uncertainty in geotag coordinates of locations and objects and allows for the possibility that objects can be visible from multiple locations in the real world.

\textbf{Data Structures.}
\emph{GESTALT} creates several data structures (inverted index, location-object concept map, object-object concept map) to facilitate spatial search over objects and locations. 
The \textit{inverted index} maps each object class to the set of locations that contain objects of that type, enabling membership queries, which prune the search space for downstream spatial searching. 
The \textit{location-object structure} treats the coordinates of the \textit{location} as the division point on the north-south and west-east axes and maps the objects for each location to the corresponding quadrant they lie in with respect to the location.
The \textit{object-object concept map} (for each location) encodes the implicit directional relationships between the objects at that location using sparse matrices that can be recursively searched to eliminate candidate locations that do not match the spatial query specification. 

\textbf{Search.}
The Search subsystem of \emph{GESTALT} enables the user to identify locations of interest based on partial knowledge about the existence and geospatial arrangement of objects at that location. 
The search process is designed for the way humans naturally conceptualize and describe locations and directions - by drawing a map.
Queries can be issued using a pictorial specification, relating objects through relative cardinal relations (like Northwest, etc.) or cardinality-invariant relations (when the user does not recall any cardinal information).
The Search subsystem supports fuzzy searching to return inexact matches when query terms are too narrow and ranking that reflects the uncertainty in an object's identification, tagging, and assignment to a location.


%\subsection{Summary of Architecture}
%The Architecture of \textit{GESTALT} is designed to be lightweight and modular. The core requirement is to automate the last-mile search problem and allow users to search for locations of interest based on partial information. 
%Achieving this outcome necessitates collecting and processing objects and locations autonomously, at scale.
%Given a large number of noisy object tags, \emph{GESTALT} performs fuzzy object assignment, allowing objects to be assigned to multiple nearby locations, thereby improving the system's recall.
%The subsequent concept mapping and spatial search processes maintain the system's precision using the relative directional relationships between query objects, extracting implicit information about the user's target location. 

%, aiming to maximize the recall of all possible objects. 
%\emph{GESTALT} currently supports ingestion of hand-labelled objects from KML Files, crowd-sourced object labels from OSM and automatically generated object tags we extract from images geolocated within our region of interest by Flickr.
%All of these data sources are fused into a common format and assigned porbabalistic scores reflecting the likelihood that the object detected is actually present in the real world at the location tagged. 
%The Data Acquisition subsystem ends when all of the data sources have been stored as JSON files, ready for ingestion by the Ownership Assignment subsystem. 

%The system accepts a JSON file of \textit{locations} and a series of JSON files of geolocated \textit{objects} both generated by the Data Acquisition subsystem. 
%instantiated into a KD-Tree with the location centroids and a nearest-neighbor search determines what location is closest to that cluster of objects, and assigns it as the parent \textit{location} for that \textit{object cluster}.
%The Ownership Assignment subsystem returns a dataframe of objects, their predicted parent locations and their respective coordinates. 


%makes no assumptions about the position of the objects relative to the location, and rather 
%The object-object concept map is sparse M x M matrix, where M is the number of objects assigned to the location, and an object at position [i,j] is the i$^th$ object from the north and the j$^th$ object from the west. 
%The concept mapping subsystem returns these three data structures - the object inverted index, the location-object indexes and the object-object matrices. 
%Each list contains the objects belonging to that location which reside in that quadrant, relative to the location centroid. 
%If the object doesn't exist at a location, there is no point in searching for its relative location.
%The Concept Mapping subsystem accepts as input the objects and their predicted parent locations. 
%Using the input dataframe, the subsystem creates two distinct data structures, intended to support different types of spatial queries. 


%including the pictorial specification of queries, leveraging the human tendency to draw scratch-maps to describe locations and directions.
%It accepts as input a user query - either through a keyword search or a pictorial query specification and the three data structures created by the concept mapping subsystem. 
%It is capable of exact and fuzzy searching. 
%The search subsystem balances precision in reducing the number of candidate locations with maximizing the recall of possible candidate locations. 
%The recall is prioritized based on the probability that they are the user's intended location. 

