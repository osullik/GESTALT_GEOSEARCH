%Humans spend a lot of time searching for things. 
%With the advent of tools like google maps and open street maps, people can search through geospatial data at a whim. 
%These tools focus on providing exact matches to queries or a list of candidate locations based on the user's query. 
%Frequently, searchers only have access to partial information. 
%Whether it has been a long time since visiting a location, they have a vague recommendation from a friend or are an investigator trying to identify a location to solve a crime- a common problem is how to find a location of interest based on partial information. 
%This project designs \textit{the \textbf{G}eospatially \textbf{E}nhanced \textbf{S}earch with \textbf{T}errain \textbf{A}ugmented \textbf{L}ocation \textbf{T}argeting (\textbf{GESTALT})}, and implements a proof-of-concept of the proposed architecture. 
%Based on a new best-case dataset developed for this project, \textit{The Swan Valley Wineries dataset}, demonstrates the functionality and utility of \textit{GESTALT} while identifying substantial opportunities for future work. 


Geographic information systems (GIS) provide users with a means to efficiently search over spatial data given certain key pieces of information, like the coordinates or exact name of a location of interest. Current GIS capabilities do not enable users to search for locations using imperfect or incomplete information easily. In these cases, GIS tools help narrow down a region of interest, but users must conduct a manual last-mile search to find the exact location of interest within that region. This typically involves the user visually inspecting many remote sensing or street-view images to identify distinct landmarks or terrain features that match the partial information provided. This step of the search process is a bottleneck. Taking inspiration from the way humans recall and search for information, we present \textit{the \textbf{G}eospatially \textbf{E}nhanced \textbf{S}earch with \textbf{T}errain \textbf{A}ugmented \textbf{L}ocation \textbf{T}argeting (\textbf{GESTALT})}, an end-to-end pipeline for extracting geospatial data, transforming it into coherent spatial relations, storing those relations, and searching over them. We contribute a new Swan Valley Wineries dataset and a proof of concept architecture that includes multiple methods for querying spatial configurations of objects, handling uncertainty in the information known about a location or object, and accounting for the fuzzy boundaries between locations.