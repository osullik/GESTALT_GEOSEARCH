\section{Search}
\label{section:search}

The core function of \textit{GESTALT} is to perform last-mile search given partial or uncertain information.
The user is assumed to know the general region of interest and some information about the objects at the location they seek.
Under these conditions, the search problem can be framed in several ways, which we describe in subsection \ref{subsection:search-method} in increasing order of complexity and utility.

\subsection{Search Query Sets}
We tested \emph{GESTALT}'s search performance by creating two sets of queries (\textit{Swan Valley ground truth queries}, and \textit{D.C. worst-case queries}). 
For reproducibility, we provide both sets of queries in the form of object names and their spatial configurations. 
We also provide the expected results set for each Swan Valley query. 
\par{The \textit{Swan Valley ground truth queries} are designed to enable comparison of precision and recall of different spatial search methods. 
To construct them, we hand-labeled the ground truth responses for 58 spatial queries on object configurations for arbitrary objects belonging to 6 wineries in the Swan Valley Wineries dataset. 
}
\par{The \textit{D.C. worst-case queries} are designed to test \emph{GESTALT}'s performance under the worst-case search conditions where the user recalls no highly distinctive features of their target location.
}
We construct these queries on the noisy Washington D.C. dataset, using the ten most common object classes as query terms (e.g., crossing, traffic signals, street lamp, bus stop).

%\vspace{-5em}
\begin{algorithm}
    \caption{Membership Search}\label{alg:search}
    \begin{algorithmic}
        \State{\textit{\textbf{Q} a list of query terms to search for}}
        \State{\textit{\textbf{II} an inverted index with objects as keys and locations as values}}
        \State{- - - - -}
        \Procedure{Search}{$Q$, $II$}
            \State{$SearchResult$ $\leftarrow$ []}
            \For{Each $P$ in $Q$}
                \State{Retrieve set of Locations $II$[$P$] and add to $SearchResult$}
            \EndFor
            \State{\textbf{return} intersection of sets of Locations in $SearchResult$}
            \EndProcedure
    \end{algorithmic}
\end{algorithm}

\begin{algorithm}[h]
    \caption{Ranked Membership Search}\label{alg:rank}
    \begin{algorithmic}
        \State{\textit{\textbf{Q} a list of query terms to search for}}
        \State{\textit{\textbf{II} an inverted index with objects as keys and locations as values}}
        \State{\textit{\textbf{SearchResult}} Set of candidate locations returned from \textbf{SEARCH()}}
        \State{- - - - -}
        \Procedure{Rank}{$SearchResult$, $Q$, $II$}%\Comment{$S$ results from $Q$ query terms in $II$}
            \State{$RankedSearchResult$ $\leftarrow$ Empty Ordered Dictionary}%\Comment{\textbf{L}ocation \textbf{R}ank}
            \For{Each location $Loc$ in $SearchResult$}%\Comment{\textbf{L}ocation}
                \State{$prob$ $\leftarrow$ 1}%\Comment{Probability location correct}
                \For{Each query point $P$ in $Q$}
                    %\State{$OP$ = Prob of Obj@Locin $II$}\Comment{$OP$ = \textbf{O}bject \textbf{P}robability}
                    \State{$prob$ $\leftarrow$ $prob$ $\times$ $P.prob$}%\Comment{$P.prob$ = object probability}
                \EndFor
                \State{$RankedSearchResult$[$Loc$] $\leftarrow$ $prob$}
            \EndFor
            \State{$RankedSearchResult \leftarrow$ Sort $RankedSearchResult$ by values}
            \State{\textbf{return} $RankedSearchResult$}
        \EndProcedure
    \end{algorithmic}
\end{algorithm}
\normalsize

\subsection{Search Method} \label{subsection:search-method}

\subsubsection{\textbf{Exact membership search}}
The simplest search function in \emph{GESTALT} (Algorithm \ref{alg:search}) takes a set of query terms representing objects the user knows are at a location and performs the appropriate look-ups and set intersections to determine which locations are a match (contain \textit{all} those objects).



\subsubsection{\textbf{Ranked membership search}}
When the exact membership search returns a large number of matches, such as for a broad query (i.e., locations that have a tree and a bench), ranking those locations can help narrow the results. Using Algorithm \ref{alg:rank}, we aggregate the confidence scores from the object tagging and ownership assignment stages to determine the overall likelihood that a given location contains the object of interest. These scores are then aggregated per location for the relevant query objects, and the final scores determine the ranking of the results. 




\subsubsection{\textbf{Fuzzy Membership Search}}
When the exact membership search returns no matches, we use a fuzzy search procedure (Algorithm \ref{alg:fuzzySearch}) to find a set of partial matches based on the most discriminative object(s) in the list of query terms. 
The most discriminative terms are emphasized since rare objects are memorable and uniquely identify locations more than common objects.

\begin{algorithm}
    \caption{Fuzzy Membership Search}\label{alg:fuzzySearch}
    \begin{algorithmic}[2]
        \State{\textit{\textbf{Q} a list of query terms to search for}}
        \State{\textit{\textbf{II} an inverted index with objects as keys and locations as values}}
        \State{- - - - -}
        \Procedure{fuzzySearch}{$Q$,$II$}
            \State{$SearchResult$ $\leftarrow$ \textbf{II.SEARCH}($Q$)}
            \If{$SearchResult$ is Empty}
                \State{$P$ $\leftarrow$ Pop most discriminative term from $Q$}
                \State{$SearchResult$ $\leftarrow$ \textbf{$II$.SEARCH}($Q$)}
                \If{$SearchResult$ is not Empty}
                    \State{Skip to Line 6}
                \Else
                    \State{$SearchResult$ $\leftarrow$ \textbf{II.SEARCH}($Q$.remove($P$)}
                \EndIf
            \EndIf
            \If{$SearchResult$ has more than 1 item}
                \State{\textbf{return} \textbf{$II$.RANK($SearchResult$, $Q$)}}
            \EndIf
        \EndProcedure
    \end{algorithmic}
\end{algorithm}
\normalsize

\subsubsection{\textbf{Spatial Search}}
\emph{GESTALT}'s core search functionality comes from the spatial search methods it enables, which rely on the spatial data structures and algorithms described in \emph{COMPASS}~\cite{Osul2023}.
The location-centric search method (Location-Object search) allows the user to query for locations based on the relative position of objects in each of the four cardinal directions, depicted in Figure \ref{fig:spatial_search_loc}.
The object-centric search method (Object-Object search) allows the user to specify the relative layout of objects relative to each other, regardless of where they are relative to the location under search (Figure \ref{fig:spatial_search_obj}.
A third form of spatial search (Cardinality Invariant Object-Object search) is also supported, which mimics the Object-Object search but removes the requirement that the query pattern be provided with the correct cardinal alignment compared to the actual pattern of objects in the real world.


\begin{figure}[h]
    \centering
    \begin{subfigure}[t]{.2\textwidth}
        \includesvg[width=\textwidth]{loc_diagram.svg}
        \caption{\small Location-object search.} 
        \label{fig:spatial_search_loc}
    \end{subfigure}
    \hfill
    \begin{subfigure}[t]{.2\textwidth}
        \includesvg[width=\textwidth]{obj_diagram.svg}
        \caption{\small Object-object search.} 
        \label{fig:spatial_search_obj}
    \end{subfigure}
    \hfill
    \caption{\textbf{Spatial search by cardinal direction to location sought (Figure \ref{fig:spatial_search_loc}) and by directional relationship between objects (Figure \ref{fig:spatial_search_obj}).}}\label{figure:spatial_search} 
\end{figure}
\normalsize


\subsection{Search Results}

\small{
\begin{table}[h!]
    \begin{center}
        \begin{tabular}{ |c|c|c|c| } 
            \hline
            Search Method & Metric & Results \\
            \hline
            \multirow{2}{18em}{Location-Object Search} & Mean Precision & $1.000$ \\
            & Mean Recall & $0.800$\\%& 
            \hline     
            \multirow{2}{18em}{Object-Object Search} & Mean Precision & $0.947$ \\ 
            &Mean Recall & $0.737$ \\
            \hline
            \multirow{2}{18em}{Object-Object Cardinality Invariant Search} & Mean Precision & $0.825$ \\ 
            &Mean Recall & $0.816$ \\
            \hline
        \end{tabular}
        \caption{Spatial search performance results across 58 ground-truth pictorial queries run on the \textit{Combined Swan Valley Wineries} dataset.} %Clustering used DBSCAN with $\epsilon = \frac{0.05}{6371}$ and MinCluster=$3$.} 
        \label{Table:GroundTruth}
    \end{center}
    \vspace{-20pt}
\end{table}
}
\normalsize
\paragraph{Search Performance.}

To evaluate \emph{GESTALT}'s search process from end to end in a noisy environment, we test it using the \textit{Swan Valley ground truth query set} and the \textit{Combined} dataset.
The results (Table \ref{Table:GroundTruth}) show that \emph{GESTALT} has a high precision and recall on all three methods of spatial search.

The cardinality-invariant version of the Object-Object search outperforms the non-invariant version in recall at the cost of precision.
The variance in performance is because rotated versions of the object configuration in some ground truth queries may happen to exist at other locations in the noisy \textit{Combined} dataset despite not being labeled in the ground truth, causing false positives.
In general, we observed that increasing the number of objects in the query reduced the false positive rate since additional constraints led to fewer random collisions with other locations not considered during the hand-labeling.
The queries tested contained between one and four objects per query, sufficient to observe reasonable precision overall, indicating the discriminative power of spatial configurations of objects as a search constraint.



\paragraph{Search Complexity.}

\small{
\begin{figure*}[h]
    \vspace{-12pt}
    \centering
    \begin{subfigure}[t]{.45\textwidth}
        \includesvg[width=\textwidth]{queryExecutionRecall.svg}
        
        %\caption{Additional terms improve the precision of results for all techniques except Loc-Obj Pictorial Querying, for which each search term adds to possible matches rather than pruning impossible matches. Users seeking to prune their search space should seek to input three objects to \emph{GESTALT}}
    \end{subfigure}
    \hfill
    \begin{subfigure}[t]{.45\textwidth} 
        \includesvg[width=\textwidth]{queryExecutionTime.svg}
        %\caption{Query time remains constant for Pictorial Obj-Obj querying after a second item because it only requires a single matrix traversal per location to match the query pattern to that candidate location. The set intersections in our Index Searches are order-optimized to execute the most discriminative sets first in queries.}    
    \end{subfigure}

    \caption{Spatial search complexity results measured in the number of candidate locations and query response times for each type of spatial query \emph{GESTALT} supports. Measured on the Washington D.C. Dataset (12,179 Locations, 91,188 objects).}\label{figure:PerformanceExperiments}\label{fig:queryExecutionRecall}        \label{fig:queryExecutionTime}
    %A suite of queries run across the D.C. Dataset (12179 Locations, 148 Distinct classes among 91188 objects) clustered using DBSCAN with $\epsilon=0.00000156961$ (roughly 50m), MinCluster=$3$ and out Fuzzy Threshold$=0.0$. We use the ten most common object classes: (eg, crossing, traffic signals, street lamp, bus stop) to simulate worst-case conditions where the user recalls almost no distinguishing features of their target location} 
    \vspace{-10pt}
\end{figure*}
}
\normalsize

%\subsubsection{Theoretical Complexity}


%\nrscomment{do some analysis on membership searches?}


%\subsubsection{Empirical Complexity}
A detailed theoretical analysis of search complexity for the three forms of spatial search is presented in \emph{COMPASS}~\cite{Osul2023}.
To assess the scalability of \emph{GESTALT} on real data, we measure query response times on the \textit{D.C. worst-case queries} and the \textit{Washington D.C. dataset} (which contains over 91,000 objects and 12,000 locations).
Figure \ref{fig:queryExecutionTime} shows these results as the number of query terms increases. 
For the three membership searches (exact, ranked, and fuzzy), the response times decrease as the number of objects specified in the query increases (i.e., as the pool of possible locations meeting the query specification narrows), following the trend in Figure \ref{fig:queryExecutionRecall} which shows the aggressive pruning that occurs as the queries become more specific.
Critically, this same effect is achieved for the object-object spatial searches, where the recursive pruning of the search space quickly eliminates any candidates that are not viable matches to the pictorial query specification.
The location-object search does not show this effect because it counts the number of objects matching the query configuration and uses that to rank candidates, so no early stopping can be done. 


%which iterates through the quadrants of each location and counts the number of objects matching the directional specification from the query,
%it does not stop early when a candidate does not match on an object, instead returning the count of matches to be used to rank all of the candidates.


%We observed an improvement in recall and reduction in precision when comparing the Cardinality Invariant form of Object-Object search to the regular form, which is expected because a spatial configuration of objects may exist for locations not labeled in the ground truth.


%A \textit{set membership problem} is the most straightforward and most efficient. Given a set of locations, each of which has a set of objects it 'owns' and a set of objects in the search term, which locations have complete coverage of the search set.

%We expect precision to be low on these queries, since the \textit{Combined} dataset includes many additional locations that were not considered during the hand-labeling.
%However, the precision we recorded is reasonable, which points to the discriminative power of spatial configurations of objects as a search constraint.


%ran 12 queries on each pictorial querying method and record the results in table \ref{Table:GroundTruth}. 
%The recall of the Loc-Obj queries tends to increase with the number of query terms, as they are additive. 
%The precision of the obj-obj queries is negatively impacted by single term queries, which returns any location with that object present.

%Fuzzy search begins with the most discriminative query term (based on counts maintained by the inverted index) and proceeds by adding query terms successively until the set intersection returns no locations matching all the criteria, at which point it walks back by one term and returns the prvious set of locations, which is the closest match.

%\subsubsection{Scalability}


%on several queries over the Washington D.C. dataset to demonstrate the scalability of the membership search functionality of \emph{GESTALT}. 

%\nrscomment{Kent, plot num locations vs time for fuzzy search runs}
