\section{Data Acquisition}
\label{section:data}

\emph{GESTALT} enables last-mile search by encoding visual and geospatial data hierarchically within a given \emph{region}, using two types of geotags: \emph{object} tags and \emph{location} tags.

\textbf{\emph{Regions}} represent a limited physical area of interest within which a last-mile search is performed. 
Regions can be defined arbitrarily and represent an administrative boundary like a city, suburb, or general geographic area.
We define regions by bounding boxes for compatibility with the OSM and Flickr APIs.

\textbf{\emph{Objects}} represent any physical entity within the region of interest. 
For example, a \textit{tree, building, lake, bridge, gate} or \textit{sign} could be an object. 
\textit{Objects} can also have attributes that provide amplifying information about them, like \textit{color, material, size, etc.}. 

\textbf{\emph{Locations}} represent physical entities that \textit{do} something, giving them a purpose beyond that of objects. 
Locations can contain meaningful groupings of objects determined by ownership, proximity, or utility. 
Examples of locations include \textit{businesses, attractions, properties, etc.}. 
\textit{Locations} have \textit{objects} associated with them, and \emph{GESTALT} enables users to query for locations given a partial set of knowledge about the objects at those locations.

\subsection{Location and Object Tags}
To maximize dataset coverage and demonstrate the flexibility of \emph{GESTALT}, we support three methods of ingesting objects and two methods of ingesting locations:
\begin{enumerate}
    \item Ingesting KML Files that contain manually annotated \textit{objects} or \textit{locations} and their coordinates.
    \item Querying the OSM API to ingest crowd-sourced \textit{object} or \textit{location} tags and their coordinates.
    \item Automatically detecting \textit{objects} in geolocated photos pulled from the Flickr API. 
\end{enumerate}

Upon ingest, each object or location is assigned a confidence score, reflecting the certainty that it exists at the coordinates tagged and is of the type annotated. 
For results reported in this paper, we adopt the rule that hand-labeled objects and locations receive a confidence score of $1.0$, OSM-labeled objects and locations receive a score of $0.75$, and automatically-detected objects receive the confidence score reported by the underlying object detection model. Table~\ref{table:DataSet} summarizes the objects and locations currently in \emph{GESTALT}.


\begin{figure}[h!]       
    \includesvg[width=0.5\textwidth]{loc_labels_plot.svg}
        \caption{The Swan Valley Wineries dataset has 6 locations with annotated objects and another 30 without manually annotated objects but with confirmed location coordinates.}
    \label{fig:loc} 
    \vspace{-20pt}
\end{figure}


\subsection{Ingest Methods}
\subsubsection{Ground Truth Hand Labeled Tags} 
We support ingesting hand-labeled objects and locations manually annotated by a trustworthy source. 
For benchmarking purposes, we curated the \emph{Swan Valley Wineries} dataset containing 31 ground truth location tags for wineries (and five for breweries) in the Swan Valley Region of Western Australia and 146 ground truth object tags associated with 6 of those wineries. 
The tags consist of an object name, its latitude \& longitude, and any descriptive markings written as key:value pairs. 
We annotated the objects manually using \textit{Google Earth Professional version 7.3}. 
The data sources include on-the-ground knowledge, manual inspection of satellite imagery, street-view imagery, and publicly available aerial photos. 
The objects tagged are representative, not exhaustive. 
Attributes of the objects (e.g., color, size, material) are also recorded. 
The object tags are aligned with their corresponding winery location in the dataset.


\subsubsection{Open Street Maps Tags}
We also support ingesting object and location tags by querying the OSM API
for crowd-sourced Geodata, including \textit{Nodes} (point data for objects and locations), \textit{ways} (line data for roads, creeks, railways, etc.), and \textit{relations} (for relationships between elements, like suburb-state).
While businesses, attractions, and other \textit{locations} are commonly annotated by the OSM community, \textit{objects} are more sparsely tagged as they are typically less interesting subject matter. 
To ingest objects, \emph{GESTALT} pulls all nodes within a given bounding box that have at least one tag (like "name," "craft," etc.) and prunes them to remove results not likely referring to objects or that lack detail to be resolved.
To ingest locations, \emph{GESTALT} uses a similar process to objects but excludes results that lack features like names, street addresses, and phone numbers. 
 

\subsubsection{Noisy Image-based Tags}
The third and most important method by which \emph{GESTALT} ingests objects is automatic object detection.
We query the Flickr API for images uploaded within a given region (bounding box) since January 1$^{st}$ 2020. 
Given those images and their EXIF metadata, the Object Detection module uses pre-trained YOLO v.8~\footnote{\href{https://github.com/ultralytics/ultralytics}{https://github.com/ultralytics/ultralytics}} to identify objects in each image from 80 classes (based on the COCO dataset~\footnote{\href{{https://cocodataset.org/}}{https://cocodataset.org/}}). 
\emph{GESTALT} stores those objects with their coordinates determined by the geolocation of the image and their confidence score determined by YOLO's confidence at the detection step.
For the Swan Valley Region, we retrieved 462 images, and for the Washington DC Region, we retrieved 4,000 images.
\small{
\begin{table}[h]
    \begin{center}
        \vspace{-10pt}
        \begin{tabular}{ |c|c|c|c|c|c| } 
            \hline
            Region & Type & Source & Label Method & \# Tags \\
            \hline
            \multirow{5}{6em}{Swan Valley\tablefootnote{BoundingBox:['115.96168231510637', '-31.90009882641578', '116.05029961853784', '-31.77307863942101']}} 
            & Objects & KML file & Hand Labeled & $146$ \\
            & Objects & OSM B-Box & Crowd-Sourced & $2466$\\
            & Objects & Flickr B-Box & Object detection & $1893$\\
            & Locations & KML file & Hand Labeled & $31$ \\
            & Locations & OSM B-Box& Crowd-Sourced & 308 \\
            \hline     
            \multirow{3}{6em}{Washington D.C.\tablefootnote{BoundingBox:['-77.120248', '38.791086', '-76.911012', '38.995732']}}             & Objects & OSM B-Box & Crowd-Sourced &$60123$ \\
            & Objects & Flickr B-Box & Object detection & $31065$ \\
            & Locations & OSM B-Box & Crowd-Sourced & $12179$ \\
            \hline
        \end{tabular}
        \caption{Location and object datasets in \emph{GESTALT}. Swan Valley region data with hand labels are part of the Swan Valley Wineries dataset we contribute. A \textit{Combined} dataset including all labeling methods for the Swan Valley region is used for some testing.}         \vspace{-25pt}
        \label{table:DataSet}
    \end{center}
\end{table}
}
\normalsize















%Creating the Swan Valley Winery dataset for this project has the benefit of yielding 31 additional nodes and associated metadata for the OSM project. 