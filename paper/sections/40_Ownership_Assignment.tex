\section{Ownership Assignment}
\label{section:ownership}

Ownership assignment enables search over locations using queries about the objects contained at or near those locations.

\subsection{Object Ownership Problem Definition}

We formulate the last-mile search problem in terms of \textit{locations}, which are searchable entities that `own' or contain any number of \textit{objects}. 
To support user queries, \emph{GESTALT} must associate objects with their parent locations.
We term this problem \textit{Object Ownership Assignment} and define it as follows.
Given a collection of locations and objects within a region, \emph{Ownership Assignment} seeks to correctly assign each object to its `parent' location. 
Ownership Assignment amounts to a clustering problem where we assign points (objects) to centroids (locations) that are determined apriori. 
The objects are not uniformly distributed across locations; some locations may not have associated tagged objects. 

Humans do not see the world in regular grid lines like on a map. Hence, the ownership assignment process is naturally inexact, and objects are `shared' between locations where appropriate.
For example, a winery may have a shed nearby that is visible both from that winery and a neighboring one. 
In this case, the object can be helpful when searching for either location. 
We address this aspect of the problem by modifying our method to allow for fuzzy object-location assignment, effectively increasing the recall. 

\subsection{Object Ownership Assignment Dataset}

We evaluate the Ownership Assignment task on two datasets: the \textit{hand-labeled} Swan Valley Wineries dataset and a \textit{Combined} dataset that includes all of the object and location tags from the Swan Valley Wineries dataset in addition to the noisy object tags ingested from Flickr and OSM.
Since we have ground truth labels for the winery locations and their objects in the Swan Valley Wineries dataset, we report both precision and recall on this dataset.
The Combined dataset presents a more realistic and challenging scenario for Ownership Assignment than the Swan Valley Wineries dataset alone, given its more extensive set of locations and a large set of object tags to assign to those locations. 
However, since the Combined dataset includes noisy tags for which we have no ground truth labels, we do not report precision. 
Instead, we measure the recall of the same set of ground-truth queries in the presence of the additional noise. 
Table \ref{table:clustering} contains the recall (and precision, where applicable) on those locations and objects.

\subsection{Object Ownership Assignment Method}

Our method for Object Ownership Assignment is outlined in Algorithm \ref{alg:OwnershipAssignment}. 
We use DBSCAN~\cite{Ester1996} to cluster the objects on the first pass and prune out noisy objects that are unlikely to be associated with any particular location.
After clustering the objects, we determine the centroids of the relevant object clusters, calculate the confidence scores in the object-cluster assignments, and then map the centroids to the nearest known location tags using a nearest-neighbor search on a KD-Tree
Objects are then added to nearby clusters using an adjustable `fuzziness' parameter.
When calculating the confidence scores, we normalize the object-centroid distances (omitting objects in the null cluster, which has no meaningful centroid and would skew the normalization). 
This confidence score (between 0 and 1) measures how far a given object is from its cluster's centroid, assigning higher scores to objects near the centroid and lower ones to objects far from the centroid. 
We take this approach rather than using a static threshold parameter (like within x distance) to account for the variety in object density of the region under search. 
We adopt the convention of assigning null cluster objects a confidence score of 0.5 since these objects are deemed noise that is not relevant for finding locations.
To further account for uncertainty in object tags, we add an adjustable parameter \textit{c} to the method, which allows for a varying degree of \textit{fuzzy assignment} of objects to clusters. 
By increasing the parameter value, we can allow for objects to be assigned to multiple clusters if they are close to multiple centroids (within \textit{$c\%$} of the largest object-centroid distance in the dataset, the same value used to normalize the confidence scores). 
\begin{algorithm}
    \caption{Object to Location Ownership Assignment Algorithm}
    \label{alg:OwnershipAssignment}
    \begin{algorithmic}
        \State{\textit{\textbf{Locs}} a list of locations and their tagged coordinates}
        \State{\textit{\textbf{Objs}} a list of objects and their coordinates}
                                                           \State{- - - - - - - - - -}
        \Procedure{ObjectOwnershipAssignment}{$Locs$, $Objs$}
            \State{$Clusters\leftarrow$ \textbf{DBSCAN}($Objs$)}            \For{Each $Cluster$ in $Clusters$ except NULL cluster}
                \State{$C \leftarrow$ Calculate centroid of $Cluster$}                \For{$Obj$ in $Cluster$}
                    \State{$Obj.prob\leftarrow 1-normalize(dist(Obj, C))$}                \EndFor
            \EndFor
            \For{Each $Obj$ in NULL Cluster}
                \State{$Obj.prob\leftarrow$ 0.5}
            \EndFor
            \For{Each $Cluster$ in $Clusters$ Except NULL Cluster}
                \State{$Cluster.Loc\leftarrow$ Closest location to $C$ in $Locs$ }            \EndFor 
        \EndProcedure
    \end{algorithmic}
\end{algorithm}
\normalsize

We run all Swan Valley Wineries experiments with DBSCAN parameter $\epsilon = \frac{0.05}{6371}$ ($\sim$50m) and DC experiments with $\epsilon = \frac{0.01}{6371}$ ($\sim$10m). For both, the MinClusterSize=$3$. We choose these values using the knowledge that the environment of interest (Swan Valley region) is semi-rural and more likely to be spread out than the urban Washington D.C. The minClusterSize is not critical to the assignment process because fuzzy assignment allows objects to be added to nearby clusters, which overcomes too fine a clustering outcome. 





\subsection{Object Ownership Assignment Results}
We show the Object-to-location assignment results for the Swan Valley Wineries and the \textit{Combined} datasets in Table \ref{table:clustering}. 
Increasing the fuzziness parameter $c$ increases the chance that objects are assigned to multiple clusters, which improves recall on the noisier \textit{Combined} dataset.
As expected, increasing the likelihood that objects are assigned to multiple clusters in the less noisy Swan Valley Wineries dataset reduces the precision but, interestingly, does not show the benefit in recall that it did for the noisier \textit{Combined} dataset.

Overall, the object ownership results show that our method handles the noisy object tags well, especially on the larger dataset. By using DBSCAN for the initial clustering, we observe a noise reduction effect as objects that do not belong to any locations are assigned to the NULL cluster (i.e., a mistagged singular object with nothing else around it for miles in any direction). A similar recall on the ground-truth labels is achieved despite adding a considerable quantity of noisy object tags (in the \textit{Combined} dataset). 

The downside of using DBSCAN in this context is that it determines the centroids of the object clusters based on the object density, and then we must map those to our known location centroids in a post-processing step. 
Ideally, a better solution would start with the centroids and cluster around them while retaining the noise reduction effects of DBSCAN.
We leave a detailed comparison of Object Ownership Assignment techniques as an interesting avenue of future study for the last-mile search problem.

\small{
\begin{table}[t]
    \begin{center}
        \begin{tabular}{ |c|c|c|c|c| } 
            \hline
            Dataset & Method & Fuzzy Param & Precision & Recall \\
            \hline
            \multirow{5}{4em}{Swan Valley Wineries} 
                & Exact & $c = 0$ & $1.0$ & $0.89$  \\ 
                & Fuzzy & $c = 2$ & $0.85$ & $0.89$  \\ 
                & Fuzzy & $c = 4$ & $0.85$ & $0.89$  \\
                & Fuzzy & $c = 6$ & $0.85$ & $0.89$  \\ 
                & Fuzzy & $c = 8$ & $0.85$ & $0.89$  \\ 
                & Fuzzy & $c = 10$ & $0.85$ & $0.89$  \\ 
            \hline
            \multirow{5}{4em}{Combined} 
                & Exact & $c = 0$ & - & $0.88$  \\ 
                & Fuzzy & $c = 2$ & - & $0.88$  \\ 
                & Fuzzy & $c = 4$ & - & $0.91$  \\ 
                & Fuzzy & $c = 6$ & - & $0.94$  \\ 
                & Fuzzy & $c = 8$ & - & $0.97$  \\ 
                & Fuzzy & $c = 10$ & - & $0.98$  \\ 
            \hline
        \end{tabular}
        \caption{ Object-to-location assignment results for Swan Valley Wineries and Combined datasets. Increasing fuzziness improves recall on the noisier \textit{Combined} dataset.}         \vspace{-2em}
        \label{table:clustering}
    \end{center}
\end{table}
}

\normalsize





















 
%The process ends when objects are mapped to their parent locations. 