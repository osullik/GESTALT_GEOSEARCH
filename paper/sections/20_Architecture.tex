\section{Architecture}
\label{section:architecture}

\begin{figure*}[t]
    \includesvg[width=\textwidth]{GESTALT_Diagram3.svg}
    \centering
    \caption[width=\textwidth]{\emph{GESTALT} consists of a data collection module, an ownership assignment module, a concept mapping module, and a search module.}
    \label{fig:architecture}
\end{figure*}


The architecture of \textit{GESTALT} (Figure \ref{fig:architecture}) covers four essential functions needed to perform last-mile search: data acquisition, object ownership assignment, concept mapping, and search. 
We describe and motivate each of them below, leaving the implementation details for sections \ref{section:data} through \ref{section:search}. 

\textbf{Data Acquisition.}
The Data Acquisition subsystem of \emph{GESTALT} leverages a variety of data sources to ingest or tag objects and locations, including hand-labeled data in Keyhole Markup Language~\footnote{\href{https://developers.google.com/kml/documentation/kml\_tut}{https://developers.google.com/kml/documentation/kml\_tut}} (KML) files, crowd-sourced object and location labels from Open Street Maps (OSM)~\cite{Haklay2008}, and automatically generated object tags extracted from Flickr images geolocated within the region of interest. 
Data Acquisition fuses heterogeneous data formats to enable common representation, storage and search.

\textbf{Ownership Assignment.}
The Ownership Assignment subsystem of \emph{GESTALT} maps the \textit{objects} to the \textit{locations} they most likely belong to using fuzzy clustering that allows objects to be assigned to multiple locations. 
\emph{GESTALT}'s assignment protocol accounts for uncertainty in geotag coordinates of locations and objects and allows for the possibility that objects can be visible from multiple locations in the real world.

\textbf{Data Structures.}
\emph{GESTALT} creates several data structures (inverted index, location-object concept map, object-object concept map) to facilitate spatial search over objects and locations. 
The \textit{inverted index} maps each object class to the set of locations that contain objects of that type, enabling membership queries, which prune the search space for downstream spatial searching. 
The \textit{location-object structure} treats the coordinates of the \textit{location} as the division point on the north-south and west-east axes and maps the objects for each location to the corresponding quadrant they lie in with respect to the location.
The \textit{object-object concept map} (for each location) encodes the implicit directional relationships between the objects at that location using sparse matrices that can be recursively searched to eliminate candidate locations that do not match the spatial query specification. 

\textbf{Search.}
The Search subsystem of \emph{GESTALT} enables the user to identify locations of interest based on partial knowledge about the existence and geospatial arrangement of objects at that location. 
The search process is designed for the way humans naturally conceptualize and describe locations and directions - by drawing a map.
Queries can be issued using a pictorial specification, relating objects through relative cardinal relations (like Northwest, etc.) or cardinality-invariant relations (when the user does not recall any cardinal information).
The Search subsystem supports fuzzy searching to return inexact matches when query terms are too narrow and ranking that reflects the uncertainty in an object's identification, tagging, and assignment to a location.









