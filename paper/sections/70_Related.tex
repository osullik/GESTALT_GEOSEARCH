\section{Related Work}
\label{section:related}
\normalsize
\emph{GESTALT} takes a human-centric approach to geospatial search, borrowing ideas from human spatial cognition, navigation, and spatial reasoning.
We address related work in human spatial reasoning and pictorial querying and then compare \emph{GESTALT} with the current closest geospatial search tool.

\subsection{Human Spatial Cognition and Reasoning}
People tend to anchor memories on objects they see while experiencing a location~\cite{Helbing2020} and use them to find it again. 
Even robots we design to do perception need visual anchoring~\cite{Oliveira2016}.
Reliance on anchoring underpins \emph{GESTALT}'s human-centric approach to last-mile spatial search wherein the goal is to retrieve locations in a region given information about their objects.
Last-mile search forms the reciprocal problem to the online game \textit{Geoguessr}~\footnote{\href{https://www.geoguessr.com/}{https://www.geoguessr.com/}}, which challenges users to determine the broad region of a photograph based on the visible objects.
Successful players of \textit{GeoGuessr} show that enough objects, even common ones, have the power to substantially prune a large search space down to just a few, or even one candidate region \cite{BernersLee2023}.
\emph{GESTALT}'s approach to geospatial search closely follows the \textit{Winthrop Method}~\cite{Keatley2021} which was developed in Northern Ireland in the 1970s to detect clandestine weapons caches.
The core principle behind this method is that humans form mental models of terrain based on objects in the environment and then use those to navigate.
Other research has shown that when revisiting locations, the objects can even serve as keys to other memories of that location~\cite{Miller2013}.
Specific to geospatial search systems, Schwering et al. synthesize the findings of multiple user studies into geospatial reasoning errors, highlighting that most humans are much better at recalling relative object positions than cardinal directions or distances between objects \cite{Schwering2014}.



\subsection{Pictorial and Spatial Pattern Matching}
\par{The spatial search functionality of \emph{GESTALT} leverages a pictorial query interface to bridge the gap between the user's mental model of objects and the geospatial data available to search. 
Pictorial query interfaces have previously allowed users to place objects and assign constraints in a \textit{drag-and-drop}~\cite{DiLoreto1996, Soffer1997, Soffer1998a, Folkers2000}, or \textit{query-by-sketch}~\cite{Schwering2014} manner. Each has its merits, but approaches that allow users to specify objects' relative positions are better suited for the \emph{GESTALT} search paradigm. 
Regardless of the interface, most query systems use an underlying query engine for \textit{spatial pattern matching} driven by \textit{set-intersection}, \textit{qualitative spatial reasoning/constraint satisfaction}, or \textit{subgraph matching}~\cite{Osul2023}.
None of these three approaches have proven to sufficiently balance scalability with result quality at a production system scale. 
\textit{GESTALT} addresses the complexity issues by successively pruning the search space, beginning with the last-mile search region, pruning out any locations that fail to contain the query terms, and then commencing the spatial search, which involves recursively pruning the search space even further, until a candidate location is determined to match or not match the configuration specified in the pictorial query.}



\subsection{Automated Geospatial Search}
In recent months, the Open Source Intelligence platform \textit{Bellingcat} released \textit{Bellingcat OSM Search}~\cite{Williams2023}, a tool also aimed at pruning search space using tagged objects. 
Using drop-downs and sliding bars, a user can specify a text-based query for OSM-tagged objects that appear within an adjustable distance from each other inside a region of interest. 
Although aimed at addressing the same last-mile search problem as \emph{GESTALT}, there are several key limitations to the Bellingcat OSM Search Tool that \emph{GESTALT} overcomes.
\begin{enumerate}
    \item Their tool only accounts for the distance between objects, not geographic configuration. \emph{GESTALT} allows users to leverage the information they remember about how the objects relate to each other spatially to improve the search result.
    \item Their tool is limited to data tagged in OSM, which is sparse for the object tags \emph{GESTALT} leverages for pruning.
    \item Their tool does not rank results, which is critical when object tags are noisy, and users are uncertain about the partial information they recall.
   \item Their tool returns 0 results if a query fails to match on every term in the region or window of interest, which is problematic when dealing with noisy, incomplete data where the likelihood of any given object being tagged is low. 
    \emph{GESTALT} uses a fuzzy matching procedure to return a partial match based on the most discriminative query term when the entire query yields no results, assuming that a partial answer is better than no answer.
\end{enumerate}